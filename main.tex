%%------------------------------------------------%%
%%  Engineer's & Master's Thesis Template         %%
%%  Copyleft by Artur M. Brodzki & Piotr Woźniak  %%
%%  Warsaw University of Technology, 2019-2021    %%
%%------------------------------------------------%%

\documentclass[
    a4paper,
    left=25mm,         % Sadly, generic margin parameter
    right=25mm,        % doesnt't work, as it is
    top=25mm,          % superseded by more specific
    bottom=25mm,         % left...bottom parameters.
    bindingoffset=5mm,  % Optional binding offset.
    nohyphenation=false % You may turn off hyphenation, if don't like. 
]{src/wut-thesis}

\langpol % Dla języka angielskiego mamy \langeng
% \facultyeiti % Wydział Elektroniki i Technik Informacyjnych
\facultymeil % Wydział Mechaniczny Energetyki i Lotnictwa
\graphicspath{{tex/img/}}         % Katalog z obrazkami.
\addbibresource{bibliografia.bib} % Plik .bib z bibliografią

\newenvironment{todo} {\color{magenta} \bfseries} {} % Formatowanie tymczasowych notatek

\begin{document}

%--------------------------------------
% Strona tytułowa
%--------------------------------------
\MasterThesis % Dla pracy inżynierskiej mamy \EngineerThesis
\instytut{Techniki Lotniczej i Mechaniki Stosowanej}
\kierunek{Lotnictwo i Kosmonautyka}
\specjalnosc{Automatyka i Systemy Lotnicze}
\title{
    System do oceny umiejętności operatorów BSP \\
    w środowisku wirtualnej rzeczywistości
}
\engtitle{ % Tytuł po angielsku do angielskiego streszczenia
    System for evaluating UAV operator performance \\
    in a virtual reality environment
}
\author{Marek Sławomir Łukasiewicz}
\album{285680}
\promotor{dr inż. Antoni Kopyt}
\date{\the\year}
\maketitle

%--------------------------------------
% Streszczenie po polsku
%--------------------------------------
\cleardoublepage % Zaczynamy od nieparzystej strony
\streszczenie 
\begin{todo}
    Streszczenie do napisania na koniec
\end{todo}
\slowakluczowe bezzałogowe statki powietrzne (BSP), interfejs człowiek maszyna, rzeczywistość wirtualna

%--------------------------------------
% Streszczenie po angielsku
%--------------------------------------
\newpage
\abstract
\begin{todo}
    ASbstract to be written at the end
\end{todo}
\keywords unmanned aerial vehicles (UAV), human-machine interface (HMI), virtual reality (VR)

%--------------------------------------
% Oświadczenie o autorstwie
%--------------------------------------
\cleardoublepage  % Zaczynamy od nieparzystej strony
\pagestyle{plain}
\makeauthorship

%--------------------------------------
% Spis treści
%--------------------------------------
\cleardoublepage % Zaczynamy od nieparzystej strony
\tableofcontents

%--------------------------------------
% Rozdziały
%--------------------------------------
\cleardoublepage % Zaczynamy od nieparzystej strony
\pagestyle{headings}

\newpage
\section{Wstęp}

Rynek bezzałogowych statków powietrznych od kilku lat wykazuje niezwykle szybki rozwój. Cywilne BSP, coraz powszechniej nazywane po prostu ,,dronami'' także w urzędowych pismach, w roku 2018 osiągnęły wartość ok. 147~mln~zł \begin{todo}źródło: Biała Księga BSP\end{todo}. Zwiększająca się liczba bezzałogowych operacji lotniczych i łatwość dostępu do sprzętu z tej dziedziny spowodowały między innymi aktualizację i ujednolicenie europejskich przepisów opisujących zasady wykonywania lotów \begin{todo}źródło: rozporządzenie w https://ulc.gov.pl/pl/drony/informacje-ogolne\end{todo}.

W wyniku wprowadzenia nowych kategorii BSP oraz uprawnień dla operatorów, konieczna była zmiana programu szkolenia. Dużą zmianą jest wprowadzenie tzw. kategorii otwartej dla lotów o niskim ryzyku. Szkolenie do uzyskania tych uprawnień jest zrealizowane w formie komputerowej, korzystając z platformy online przygotowanej przez Urząd Lotnictwa Cywilnego \begin{todo}źródło: https://ulc.gov.pl/pl/drony/informacje-ogolne\end{todo}. Wskazuje to, że można się spodziewać rosnącego udziału symulatorów w procesie szkolenia, podobnie jak ma to miejsce w przypadku komercyjnego lotnictwa załogowego.

Przed 31 grudnia 2020 roku, kiedy zostały wprowadzone nowe przepisy, szkolenie do wszystkich kategorii świadectwa kwalifikacji operatora bezzałogowego statku powietrznego (UAVO) odbywało się w formie tradycyjnej. Część teoretyczna w formie wykładów, po której następowały loty z instruktorem. Jedynym wskaźnikiem postępów kursanta, oraz kryterium zdania egzaminu praktycznego była opinia instruktora lub egzaminatora. Celem niniejszej pracy jest próba przygotowania systemu który oprócz symulacji dynamiki BSP umożliwia przynajmniej częściową implementację oceny jakości pilotażu.

Obecnie dostępne symulatory kierowane są dla osób zainteresowanych zdalnie sterowanymi modelami latającymi. To przeznaczenie powoduje że ich użyteczność w szkoleniu UAVO jest bardzo ograniczona. Często wśród wielu modeli samolotów i śmigłowców znajdują się zaledwie jeden lub dwa wielowirnikowce. Ponadto otoczenie w którym wykonywany jest lot zawiera jedynie pas startowy i okazjonalnie inne przeszkody \begin{todo}przykład: RealFlight\end{todo}, natomiast egzamin praktyczny wymaga m.~in. wykonywania konkretnych manewrów w odniesieniu do słupków drogowych.


%--------------------------------------------
% Literatura
%--------------------------------------------
\cleardoublepage % Zaczynamy od nieparzystej strony
\printbibliography

%--------------------------------------------
% Spisy (opcjonalne)
%--------------------------------------------
\newpage
\pagestyle{plain}

% Wykaz symboli i skrótów.
% Pamiętaj, żeby posortować symbole alfabetycznie
% we własnym zakresie. Ponieważ mało kto używa takiego wykazu,
% uznałem, że robienie automatycznie sortowanej listy
% na poziomie LaTeXa to za duży overkill.
% Makro \acronymlist generuje właściwy tytuł sekcji,
% w zależności od języka.
% Makro \acronym dodaje skrót/symbol do listy,
% zapewniając podstawowe formatowanie.
% //AB
\vspace{0.8cm}
\acronymlist
\acronym{BSP}{bezzałogowy statek powietrzny}

\listoffigurestoc     % Spis rysunków.
\vspace{1cm}          % vertical space
\listoftablestoc      % Spis tabel.
\vspace{1cm}          % vertical space
\listofappendicestoc  % Spis załączników

% Tabele i wykresy w załączniku nie są brane pod uwagę w spisie tabel i wykresów.
% Ich numeracja jest zresetowana.
\captionsetup[figure]{list=no}
\captionsetup[table]{list=no}

% Załączniki
% \newpage
% \appendix{Nazwa załącznika 1}
% \lipsum[1-4]

% Używając powyższych spisów jako szablonu,
% możesz tu dodać swój własny wykaz bądź listę,
% np. spis algorytmów.

\end{document} % Dobranoc.
