\newpage
\section{Wstęp}

Rynek bezzałogowych statków powietrznych od kilku lat wykazuje niezwykle szybki rozwój. Cywilne BSP, coraz powszechniej nazywane po prostu ,,dronami'' także w urzędowych pismach, w roku 2018 osiągnęły wartość ok. 147~mln~zł \cite{bialaksiega2019}. Zwiększająca się liczba bezzałogowych operacji lotniczych i łatwość dostępu do sprzętu z tej dziedziny spowodowały między innymi aktualizację i ujednolicenie europejskich przepisów opisujących zasady wykonywania lotów \cite{eu-945-2019}.

W wyniku wprowadzenia nowych kategorii BSP oraz uprawnień dla operatorów, konieczna była zmiana programu szkolenia. Dużą zmianą jest wprowadzenie tzw. kategorii otwartej dla lotów o niskim ryzyku. Szkolenie do uzyskania tych uprawnień jest zrealizowane w formie komputerowej, korzystając z platformy online przygotowanej przez Urząd Lotnictwa Cywilnego \cite{ulc2019}. Wskazuje to, że można się spodziewać rosnącego udziału symulatorów w procesie szkolenia, podobnie jak ma to miejsce w przypadku komercyjnego lotnictwa załogowego.

Przed 31 grudnia 2020 roku, kiedy zostały wprowadzone nowe przepisy, szkolenie do wszystkich kategorii świadectwa kwalifikacji operatora bezzałogowego statku powietrznego (UAVO) odbywało się w formie tradycyjnej. Część teoretyczna w formie wykładów, po której następowały loty z instruktorem. Jedynym wskaźnikiem postępów kursanta, oraz kryterium zdania egzaminu praktycznego była opinia instruktora lub egzaminatora. Celem niniejszej pracy jest próba przygotowania systemu który oprócz symulacji dynamiki BSP umożliwia przynajmniej częściową implementację oceny jakości pilotażu.

Obecnie dostępne symulatory kierowane są dla osób zainteresowanych zdalnie sterowanymi modelami latającymi. To przeznaczenie powoduje że ich użyteczność w szkoleniu UAVO jest bardzo ograniczona. Często wśród wielu modeli samolotów i śmigłowców znajdują się zaledwie jeden lub dwa wielowirnikowce. Ponadto otoczenie w którym wykonywany jest lot zawiera jedynie pas startowy i okazjonalnie inne przeszkody, natomiast egzamin praktyczny wymaga m.~in. wykonywania konkretnych manewrów w odniesieniu do słupków drogowych.

\begin{todo}
Ogólnie o symulatorach lotu załogowych - wsadza w kontekst, jaka to dziedzina nauki. Rozszerzyć na symulatory innych rzeczy - koparki, samochody, mechanika.

Podrozdział o grach komputerowych - narzędzia gamedev przechodzą do przemysłu

Przejść do symulatorów BSP - wskazać przykłady, w tym przeglądzie opisać jak nie spełniają wymagań.
\end{todo}
