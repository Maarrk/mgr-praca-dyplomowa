\newpage
\section{Wymagania wobec systemu}
W ramach Ośrodka Badań Lotniczych Politechniki Warszawskiej trwają prace nad przygotowaniem nowego sposobu szkolenia operatorów BSP. Planowane rozwiązanie zakłada wykorzystanie trzech trybów wykonywania zadań:
\begin{enumerate}
  \item[tryb wirtualny]: wykorzystujący symulację komputerową lotu
  \item[tryb mieszany]: lot rzeczywistym BSP pomiędzy wirtualnymi przeszkodami
  \item[tryb rzeczywisty]; lot BSP pomiędzy rzeczywistymi przeszkodami
\end{enumerate}
Wyróżniającą to podejście cechą jest wprowadzenie trybu mieszanego. W tym wariancie szkolenia operator sterując rzeczywistym BSP ma omijać przeszkody które są tylko symulowane i wyświetlane. Dzięki temu umiejętności pilotażu są nabywane z wykorzystaniem docelowego statku powietrznego, natomiast ryzyko uszkodzenia sprzętu w wyniku kolizji jest dużo niższe dzięki zastosowaniu niematerialnych przeszkód. Preferowanym sposobem realizacji tych założeń jest przygotowanie jednego narzędzia które może być wykorzystywane we wszystkich trzech trybach ćwiczenia pilotażu. Dzięki temu dla raz opracowanego zadania będzie można stopniowo zwiększać poziom ryzyka przez kolejne etapy wymienione powyżej.

Tryb wirtualny oraz mieszany stawiają wymagania wobec wizualizacji oferowanej przez system. Ponieważ orientacja przestrzenna oraz ocena odległości są ważnymi umiejętnościami w obsłudze statków powietrznych wymagane jest aby system obsługiwał gogle wirtualnej rzeczywistości (VR), które przez stereowizję tworzą pozory trójwymiarowego obrazu. Ponadto, implementacja trybu mieszanego wymaga zastosowania wyświetlacza do rzeczywistości rozswzerzonej (AR). Jest to bardzo podobne urządzenie do okularów VR opisywanych we wstępie, jednak wykorzystujące przezierne wyświetlacze. Dzięki temu, możliwe jest łączenie obrazu widzianego bezpośrednio przez użytkownika z elementami syntetyzowanymi przez komputer. W tym zastosowaniu, taki wyświetlacz umożliwia widoczność wirtualnych przeszkód przy jednoczesnej obserwacji BSP.

Podstawowym przeznaczeniem przygotowywanego systemu jest ułatwienie prowadzenia eksperymentów na temat szkolenia operatorów BSP. Z tego powodu kluczową cechą jest elastyczność i łatwość rozszerzania systemu o nowe funkcje oraz scenariusze testowe. Interfejs użytkownika powinien umożliwiać łatwe tworzenie nowych otoczeń oraz zadań dla operatora. Także funkcjonalność automatycznej oceny operatora powinna umożliwiać użytkownikowi aplikacji formułowanie własnych kryteriów oceny. W zależności od wykonywanego eksperymentu ocena może korzystać z podstawowych parametrów lotu takich jak położenie, prędkość, orientacja przestrzenna. Ponadto aplikacja ma umożliwiać wprowadzenie wzorcowej trasy poprawnie wykonanego ćwiczenia, a następnie oceniać przelot na podstawie odchyłek od niej.

Do poprawnego działania trybu mieszanego niezbędnę będzie połączenie z wykorzystywanym statkiem powietrznym. Ponieważ wtedy przeszkody są wirtualne i nie wpływają w żaden sposób na lot statku powietrznego, jedynym sposobem wykrycia kolizji jest stałe śledzenie położenia BSP względem poszczególnych symulowanych obiektów. Korzystając z tego samego połączenia, możliwe będzie tez stosowanie funkcji oceniających dla lotów pomiędzy rzeczywistymi przeszkodami jeśli zostaną poprawnie opisane w symulacji. W locie w trybie wirtualnym z braku fizycznego statku powietrznego którego stan można by śledzić, konieczne jest prowadzenie symulacji jego dynamiki. Oprócz tego, aby nawyki nabyte w trakcie szkolenia w pełnej symulacji miały przełożenie na sterowanie fizycznym statkiem powietrznym, ważne jest aby wykorzystywać takie same sterownice jak w rzeczywistości. Z tego powodu niezbędna jest możliwość sterowania symulacją wykorzystując nadajnik zdalnego sterowania używany do rzeczywistych lotów.

Podsumowując, sformułowano następujące wymagania:
\begin{itemize}
  \item wyświetlanie symulacji za pomocą okularów VR
  \item wyświetlanie symulacji za pomocą okularów AR
  \item możliwość implementacji różnych zadań
  \item możliwość oceniania operatora na podstawie róznych parametrów lotu
  \item możliwość oceny operatora za pomocą różnych funkcji
  \item połączenie na żywo z latającym BSP
  \item możliwość lotu symulowanym BSP za pomocą nadajnika zdalnego sterowania.
\end{itemize}
