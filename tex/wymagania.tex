\newpage
\section{Wymagania wobec systemu}

Podstawowym przeznaczeniem przygotowywanego systemu jest ułatwienie prowadzenia eksperymentów na temat szkolenia operatorów BSP korzystając z aplikacji komputerowych. Z tego powodu kluczową cechą jest elastyczność i łatwość rozszerzania systemu o nowe funkcje oraz scenariusze testowe. Interfejs użytkownika powinien umożliwiać łatwe tworzenie nowych otoczeń oraz zadań dla operatora. Ponieważ orientacja przestrzenna oraz ocena odległości są ważnymi umiejętnościami w obsłudze statków powietrznych \begin{todo}(¿jakie dać do tego źródło?)\end{todo}, wymagane jest aby system obsługiwał gogle wirtualnej rzeczywistości (VR), które przez stereowizję tworzą pozory trójwymiarowego obrazu.

Dodatkowym wymogiem postawionym przez użytkownika aplikacji jest możliwość zastosowania go w trybie mieszanej rzeczywistości (AR). W tym wariancie szkolenia operator sterując rzeczywistym BSP ma omijać przeszkody które są wirtualne. Dzięki temu umiejętności pilotażu są nabywane z wykorzystaniem docelowego statku powietrznego, natomiast ryzyko uszkodzenia sprzętu w wyniku kolizji jest dużo niższe dzięki zastosowaniu niematerialnych przeszkód.

Funkcjonalność automatycznej oceny operatora powinna umożliwiać użytkownikowi aplikacji formułowanie własnych kryteriów oceny. W zależności od wykonywanego eksperymentu ocena może korzystać z podstawowych parametrów lotu takich jak położenie, prędkość, orientacja przestrzenna. Ponadto aplikacja ma umożliwiać wprowadzenie wzorcowej trasy poprawnie wykonanego ćwiczenia, a następnie oceniać przelot na podstawie odchyłek od niej.

\begin{todo}
Podsumować w punktach, przeredagować żeby były w grupach.

Opisać technologie VR i AR w podrozdziale. Wskazać różnice, dlaczego AR tu będzie dobry.

Wspomnieć o trójstopniowym szkoleniu - wirtualne, mieszane, rzeczywiste. Nawiązać do OBLOT.
\end{todo}